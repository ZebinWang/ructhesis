%\iffalse meta-comment
% Copyright (C) 2003--2015 Zebin Wang
% --------------------------------------------------------------------------
%
% This work may be distributed and/or modified under the
% conditions of the LaTeX Project Public License, either
% version 1.3c of this license or (at your option) any later
% version. This version of this license is in
%    http://www.latex-project.org/lppl/lppl-1-3c.txt
% and the latest version of this license is in
%    http://www.latex-project.org/lppl.txt
% and version 1.3 or later is part of all distributions of
% LaTeX version 2005/12/01 or later.
%
% This work has the LPPL maintenance status `maintained'.
%
% The Current Maintainers of this work is Zebin Wang.
%
% This work consists of the files RUCThesis.dtx and RUCThesis.ins
% and the derived file RUCThesis.cls.
%
% --------------------------------------------------------------------------
%\fi
%\iffalse
%<*driver>
\ProvidesFile{ructhesis.dtx}[2015/12/23 V.1.0.0]
\documentclass[10pt]{ltxdoc}
\usepackage[UTF8]{ctex}
\usepackage{tikz}
\usepackage{listings}
\usepackage{xcolor}
\usepackage{hyperref}
\usetikzlibrary{trees}
\EnableCrossrefs
\CodelineIndex
\RecordChanges
\begin{document}
	\DocInput{\jobname.dtx}
\end{document}
%</driver>
%\fi
% 
% \DoNotIndex{\begin,\end,\begingroup,\endgroup}
% \DoNotIndex{\ifx,\ifdim,\ifnum,\ifcase,\else,\or,\fi}
% \DoNotIndex{\let,\def,\xdef,\newcommand,\renewcommand}
% \DoNotIndex{\expandafter,\csname,\endcsname,\relax,\protect}
% \DoNotIndex{\Huge,\huge,\LARGE,\Large,\large,\normalsize}
% \DoNotIndex{\small,\footnotesize,\scriptsize,\tiny}
% \DoNotIndex{\normalfont,\bfseries,\slshape,\interlinepenalty}
% \DoNotIndex{\hfil,\par,\hskip,\vskip,\vspace,\quad}
% \DoNotIndex{\centering,\raggedright}
% \DoNotIndex{\c@secnumdepth,\@startsection,\@setfontsize}
% \DoNotIndex{\ ,\@plus,\@minus,\p@,\z@,\@m,\@M,\@ne,\m@ne}
% \DoNotIndex{\@@par,\DeclareOperation,\RequirePackage,\LoadClass}
% \DoNotIndex{\AtBeginDocument,\AtEndDocument,\\,\,}
%
% \IndexPrologue{\section*{索引}%
%    \addcontentsline{toc}{section}{索~~~~引}}
% \GlossaryPrologue{\section*{修改记录}%
%    \addcontentsline{toc}{section}{修改记录}}
% \title{中国人民大学\LaTeX\ 论文模板}
% \author{王泽斌\\ {\tt me@zebinwang.com}}
% \date{2015.12.23}
% \maketitle
% \begin{abstract}\noindent
%   RUCThesis文档类旨在建立符合中国人民大学《本科论文指导手册》和《研究生学位论 文及其摘要的撰写和印制要求》的 \LaTeX\ 学位论文模板。
% \end{abstract}
% \vskip2cm
% \def\abstractname{免责声明}
% \begin{abstract}
% \noindent
% \begin{enumerate}
% \item 本模板的发布遵守 \LaTeX{} Project Public License,使用前请认真阅读协议内
%   容。
% \item 任何个人或组织以本模板为基础进行的使用、修改、拓展而产生的一切后果自行承担。
% \end{enumerate}
% \end{abstract}
% \clearpage
% \begin{multicols}{2}[
%   \setlength{\columnseprule}{.4pt}
%   \setlength{\columnsep}{18pt}]
%   \tableofcontents
% \end{multicols}
% \clearpage
% \section{安装}
% \subsection{下载}
% 目前RUCThesis提供如下两种下载方式:
% \href{https://github.com/ZebinWang/ructhesis}{GitHub}、\href{}{CTAN}
% \subsection{文件组成}
%下图为本模板的文件结构。其中\tt chap\rm 为章节目录,\tt figure\rm 为图像文件夹其中名称中有logo字样的为模板必要的图像,\tt ref\rm 为参考文献文件夹。其余所有文件都已标注在图中。
% \begin{figure}[htbp]
% \centering
% \tikzstyle{every node}=[anchor=west]
% \begin{tikzpicture}[%
%  grow via three points={one child at (0.5,-0.5) and
%  two children at (0.5,-0.5) and (0.5,-1.0)},
%  edge from parent path={(\tikzparentnode.south) |- (\tikzchildnode.west)}]
%  \node {RUCThesis}
% child { node {chap}
%    	child { node {chapter1.tex\quad \% 章节文件}}
%      	child { node {...}}
%      	child { node {appendix\_{}1.tex\quad \% 附录}}} 
% child [missing] {} 
% child [missing] {} 
% child [missing] {}
% child { node {cover.tex\quad \% 封面文件}}
% child { node {figures}
%	child { node {logo.pdf\quad \% 图像}}
%	child { node {...}}}
% child [missing] {} 
% child [missing] {} 
% child { node {format}
%    	child { node {acknowledge.tex\quad \% 致谢}}
%      	child { node {authorization.tex\quad \% 授权书影印件}}
%      	child { node {cabstractpage.tex\quad \% 中文摘要}}
%	child { node {eabstractpage.tex\quad \% 英文摘要}}
%      	child { node {Originality.tex\quad \% 独创性声明}}} 
% child [missing] {} 
% child [missing] {} 
% child [missing] {}
% child [missing] {} 
% child [missing] {} 
% child { node {main.tex\quad \% 主文件}}
% child { node {ref}
%	child { node {ruc.bst\quad \% 参考文献样式}}
%	child { node {yourbib.bib\quad \% 参考文献数据库}}}
% child [missing] {} 
% child [missing] {} 
% child { node {ructhesis.cls\quad \% RUCThesis文档类}}
% child { node {ructhesis.dtx\quad \% RUCThesis源代码}}
% child { node {ructhesis.ins\quad \% RUCThesis安装文件}};
% \end{tikzpicture}
% \caption{RUCThesis文件目录} 
% \end{figure}
% \subsection{生成模板}
% 如果对本模板机理不感兴趣的读者可以略过这里。模板源文件是\tt ructhesis.ins\rm 和\tt ructhesis.dtx\rm 文件,在终端中执行如下代码:
%\\ \tt \$ latex ructhesis.ins
%\\\rm 可以得到ructhesis.cls文件,继续执行如下代码:
%\\ \tt \$ xelatex ructhesis.dtx
%\\ \tt \$ makeindex -s gind.ist -o ructhesis.ind ructhesis.idx
%\\ \tt \$ makeindex -s gglo.ist -o ructhesis.gls ructhesis.glo
%\\ \tt \$ xelatex ructhesis.dtx
%\\ \tt \$ xelatex ructhesis.dtx
%\\\rm 可以得到本文档,这里不再赘述。
%\subsection{使用模板}
%如果你的模板文件中已经有\tt ructhesis.cls\rm 文件可以直接略过上一步。
%\StopEventually{\PrintChanges\PrintIndex}
% \section{程序代码}
% \subsection{文档类及选项定义}
%    \begin{macrocode}
%<*cls>
\NeedsTeXFormat{LaTeX2e}
\ProvidesClass{ructhesis}[2015/12/01 v1.0.0]
\newif\ifruc@bachelor
\newif\ifruc@master
\newif\ifruc@doctor
\newif\ifruc@promaster
\newif\ifruc@shuji
%    \end{macrocode}
% \begin{macro}{bachelor}  
% \begin{macro}{master}  
% \begin{macro}{promaster}
% \begin{macro}{doctor}  本科、硕士、专业硕士、博士选项
%    \begin{macrocode}
\DeclareOption{bachelor}{\ruc@bachelortrue}
\DeclareOption{master}{\ruc@mastertrue}
\DeclareOption{promaster}{\ruc@promastertrue}
\DeclareOption{doctor}{\ruc@doctortrue}
%    \end{macrocode}
% \end{macro}	
% \end{macro}	
% \end{macro}	
% \end{macro}	
% \begin{macro}{shuji}  这个选项只能在\tt cover.tex\rm 中使用,代表封皮是否打印书脊。
%    \begin{macrocode}
\DeclareOption{shuji}{\ruc@shujitrue}
\DeclareOption*{\PassOptionsToClass{\CurrentOption}{ctexbook}}
\ProcessOptions\relax
\ifruc@doctor\LoadClass[UTF8,zihao=-4,twoside,openright,fancyhdr]{ctexbook}
\else\LoadClass[UTF8,zihao=-4,oneside,openany,fancyhdr]{ctexbook}\fi
%    \end{macrocode}
% \end{macro}	
% \subsection{宏}
%    \begin{macrocode}
\def\thesiscode#1{\gdef\@thesiscode{#1}}
\def\sign#1{\gdef\@sign{#1}}
\def\esign#1{\gdef\@esign{#1}}
\def\title#1{\gdef\@title{#1}}
\def\subtitle#1{\gdef\@subtitle{#1}}
\def\author#1{\gdef\@author{#1}}
\def\school#1{\gdef\@school{#1}}
\def\field#1{\gdef\@field{#1}}
\def\grade#1{\gdef\@grade{#1}}
\def\studentid#1{\gdef\@studentid{#1}}
\def\advisor#1{\gdef\@advisor{#1}}
\def\score#1{\gdef\@score{#1}}
\def\date#1{\gdef\@date{#1}}
\def\keywords#1{\gdef\@keywords{#1}}
\def\etitle#1{\gdef\@etitle{#1}}
\def\covertitle#1{\gdef\@covertitle{#1}}
\def\keywordzh#1{\gdef\@keywordzh{#1}}
\def\keyworden#1{\gdef\@keyworden{#1}}
\def\doctorsign{博士学位论文}
\def\doctoresign{DOCTORAL DISSERTATION}
\def\mastersign{硕士学位论文}
\def\masteresign{THESIS OF MASTER DEGREE}
\def\promastersign{专业硕士学位论文}
\def\promasteresign{THESIS OF PROFESSION MASTER DEGREE}
\def\bachelorsign{本科毕业论文}
\def\bacheloresign{THESIS OF BACHELOR DEGREE}
\def\cabstractpagesign{摘要}
\def\eabstractpagesign{Abstract}
\def\acknowledgesign{致谢}
\newcommand{\RUCThesis}{{\tt R\kern-.107em\lower.5ex\hbox{U}\kern-.1em CThesis}}
%    \end{macrocode}
% \subsection{页面布局}
%    \begin{macrocode}
\ifruc@bachelor
\RequirePackage[top=25mm,left=25mm,bottom=25mm,right=20mm,footskip=10mm]{geometry}
\setlength{\topskip}{10mm}
\else
\RequirePackage[top=45mm,left=35mm,bottom=40mm,right=30mm,headsep=20mm]{geometry}
\fi
%    \end{macrocode}
% \subsection{PDF}
%    \begin{macrocode}
\RequirePackage{hyperref}
\AtBeginDocument{
\hypersetup{
	pdftitle={\@title},
	pdfauthor={\@author},
	pdfsubject={中国人民大学学位论文}}
}
%    \end{macrocode}
% \subsection{字体}
%    \begin{macrocode}
\setmainfont{Times New Roman} 
\setsansfont{Arial}
\setmonofont{Courier New}
\xeCJKsetup{AutoFakeSlant={true}} 
\setCJKmainfont{SimSun} 
\setCJKsansfont{SimHei} 
\setCJKmonofont{FangSong} 
%    \end{macrocode}
% \subsection{行距}
% \begin{macro}{\rowspace}  全文行距调整
%    \begin{macrocode}
\newcommand{\rowspace}{%
	\setlength{\lineskiplimit}{2.625bp}
	\setlength{\lineskip}{2.625bp}
	\ifruc@bachelor
		\linespread{1.25}\selectfont
	\else
		\linespread{1.35}\selectfont
	\fi
}
%    \end{macrocode}
% \end{macro}	
% \subsection{空白页样式}
% \begin{macro}{\cleardoublepage} 空白页样式设为empty
%    \begin{macrocode}
\let\ruc@cleardoublepage\cleardoublepage
\newcommand{\ruc@clearemptydoublepage}{%
\clearpage{\pagestyle{empty}\ruc@cleardoublepage}}
\let\cleardoublepage\ruc@clearemptydoublepage
%    \end{macrocode}
% \end{macro}	
% \subsection{页眉页脚}
% \noindent 这里之所以定义了这么多,完全是为了处理本科的要求......
%    \begin{macrocode}
\pagestyle{fancy}
	\fancyhf{}
	\lhead{}
	\rhead{}
	\ifruc@bachelor
		\chead{\includegraphics[width=4.13cm,height=0.8452cm]{figures/logo.pdf}}
		\fancyfoot[CO,CE]{ 第 \thepage 页}
	\else
		\chead{\zihao{5}\@title}
		\if@twoside
			\fancyfoot[RO,LE]{\thepage}
		\else
			\fancyfoot[RO,RE]{\thepage}
		\fi
	\fi
\fancypagestyle{headings}{
	\fancyhf{}
	\lhead{}
	\rhead{}
	\ifruc@bachelor
		\chead{\includegraphics[width=4.13cm,height=0.8452cm]{figures/logo.pdf}}
		\fancyfoot[CO,CE]{\thepage}
	\else
		\chead{}\renewcommand*{\headrulewidth}{0bp}
		\if@twoside
			\fancyfoot[RO,LE]{\thepage}
		\else
			\fancyfoot[RO,RE]{\thepage}
		\fi
	\fi}
\fancypagestyle{myheadings}{
	\fancyhf{}
	\lhead{}
	\rhead{}
	\chead{\includegraphics[width=4.13cm,height=0.8452cm]{figures/logo.pdf}}
	\fancyfoot[LO,LE]{}
	\fancyfoot[RO,RE]{}
	\fancyfoot[CO,CE]{}}
\fancypagestyle{plain}{
	\fancyhf{}
	\lhead{}
	\rhead{}
	\ifruc@bachelor
		\chead{}\renewcommand*{\headrulewidth}{0pt} 
		\fancyfoot[CO,CE]{ 第 \thepage 页}
	\else
		\chead{}
		\renewcommand*{\headrulewidth}{0pt} 
		\if@twoside
			\fancyfoot[RO,LE]{\thepage}
		\else
			\fancyfoot[RO,RE]{\thepage}
		\fi
	\fi}
%    \end{macrocode}
% \subsection{扉页}
% \begin{macro}{\maketitle} 插入扉页
%    \begin{macrocode}
\RequirePackage{graphicx}
\newif\if@subtitle
\ifruc@bachelor
\renewcommand{\maketitle}{ 
	\newgeometry{top=25mm,left=25mm,bottom=40mm,right=20mm,footskip=0mm,a4paper}
	\setlength{\topskip}{5mm}
	\thispagestyle{myheadings}
	\linespread{1.5}\selectfont
	{\hfill\zihao{-4}\sf\@thesiscode
	\begin{center}\zihao{1}\rule[0mm]{0mm}{25mm}\@sign\par\vspace{15mm}
	\@title\par
	\zihao{2}\@subtitletrue\par\vskip\stretch{1}\zihao{3}
	作\qquad 者:\underline{\makebox[90mm]\@author}\hfill\par 
	学\qquad 院:\underline{\makebox[90mm]\@school}\hfill\par
	专\qquad 业:\underline{\makebox[90mm]\@field}\hfill\par
	年\qquad 级:\underline{\makebox[90mm]\@grade}\hfill\par
	学\qquad 号:\underline{\makebox[90mm]\@studentid}\hfill\par
	指导教师:\underline{\makebox[90mm]\@advisor}\hfill\par
	论文成绩:\underline{\makebox[90mm]\@score}\hfill\par
	日\qquad 期:\underline{\makebox[90mm]\@date}\hfill
	\end{center}}
	\restoregeometry
	\rowspace%本科全文行距	
}
\else
\renewcommand{\maketitle}{ 
	\pagestyle{empty}
	\begin{center}
	\includegraphics[width=7.6cm,height=1.474cm]{figures/clogo.pdf}\\
	\sf\zihao{1}\ziju{0.4}\@sign\par
	\tt\zihao{3}
	\vspace{25mm}\ziju{0}\linespread{1.5}\selectfont
	(中文题目)\underline{\parbox[b]{110mm}\@title}\hfill\par\vspace{3mm}
	(英文题目)\underline{\parbox[b]{110mm}\@etitle}\hfill\par
	\vfill\ziju{0.65}
	作者学号:\underline{\makebox[80mm]\@studentid}\hfill\par
	作者姓名:\underline{\makebox[80mm]\@author}\hfill\par
	所在学院:\underline{\makebox[80mm]{\ziju{0.2}\@school}}\hfill\par
	专业名称:\underline{\parbox[b]{80mm}{\centering\ziju{0}\@field}}\hfill\par\ziju{0.65}
	导师姓名:\underline{\makebox[80mm]\@advisor}\hfill\par\ %
	\ziju{0.25}
	论文主题词:\underline{\parbox[b]{80mm}{\vspace*{1pt}\centering\ziju{0}\@keywords}}
	\hfill\par\ziju{0}
	论文提交日期:\underline{\makebox[80mm]\@date}\hfill
	\end{center}
	\rowspace%研究生全文行距
	\clearpage
	\if@twoside
		\thispagestyle{empty}
		\vspace*{\stretch{1}}
		{\zihao{5}
		\definecolor{light-gray}{gray}{0.86}
		\noindent
		\textcolor{light-gray}{
		Typeset by \LaTeXe{}\\
		With package \tt{\RUCThesis}\\
		}}
	\cleardoublepage
	\fi	
}
\fi
%    \end{macrocode}
% \end{macro}	
% \subsection{中文摘要}
% \begin{macro}{abstractzh} 中文摘要环境
%    \begin{macrocode}
\newenvironment{abstractzh}
{\clearpage\chapter*{\cabstractpagesign}\vspace*{-5mm}
	\pagenumbering{Roman}\zihao{-4}\rm}
{\par\vspace*{7mm}\noindent\sf\zihao{-4}关键词:
	\rm\zihao{-4}\@keywordzh
	\thispagestyle{headings}}
%    \end{macrocode}
% \end{macro}	
% \subsection{英文摘要}
% \begin{macro}{abstracten} 英文摘要环境
%    \begin{macrocode}
\newenvironment{abstracten}
{\clearpage
	\ifruc@bachelor\linespread{2}\selectfont
	\chapter*{\bf{\eabstractpagesign}}\vspace*{-5mm}
	\else\chapter*{\sf{\eabstractpagesign}}\vspace*{-5mm}\fi
	\zihao{-4}\rm}
{\par\vspace*{7mm}\noindent
	\zihao{-4}\textbf{Key Words : }
	\rm\zihao{-4}\@keyworden
	\thispagestyle{headings}\rowspace}
%    \end{macrocode}
% \end{macro}	
% \subsection{致谢}
% \begin{macro}{\acknowledge} 致谢环境
%    \begin{macrocode}
\newenvironment{acknowledge}
{\chapter*{\acknowledgesign}\vspace*{-5mm}
	\addcontentsline{toc}{chapter}{\acknowledgesign}\zihao{-4}\rm}
{\thispagestyle{plain}}
%    \end{macrocode}
% \end{macro}	
% \subsection{授权}
% \begin{macro}{\authorization} 插入授权影印件
%    \begin{macrocode}
\newcommand{\authorization}[1]
{\ifruc@doctor\cleardoublepage\else\clearpage\fi
\newgeometry{top=0mm,left=0mm,bottom=0mm,right=0mm,
text={\paperwidth,\paperheight},marginparwidth=0mm}\hspace{-9mm}
\thispagestyle{empty}
\includegraphics[width=\paperwidth-1mm ,totalheight=\paperheight-1mm]{#1}
\restoregeometry}
%    \end{macrocode}
% \end{macro}	
% \subsection{独创性声明}
% \begin{macro}{\originality} 插入独创性声明
%     \begin{macrocode}	
\newcommand{\originality}{
\chapter*{\zihao{-2}\heiti 独创性声明}
{\tt\zihao{-4}本人郑重声明:所呈交的论文是我个人在导师的指导下进行的研究工作及取得的研究成果。尽我所知,除了文中特别加以标注和致谢的地方外,论文中不包含其他人已经发表或撰写的研究成果,也不包含为获得中国人民大学或其他教育机构的学位或证书所使用过的材料。与我一同工作的同志对本研究所做的任何贡献均已在论文中作了明确的说明并表示了谢意。\par}
\vskip 10mm\hfill 
论文作者:\rule[-1ex]{30mm}{0.25pt}\qquad 
日\qquad 期:\rule[-1ex]{30mm}{0.25pt}\par
\vfil\centerline{\heiti\zihao{-2}关于论文使用授权的说明}
\vskip 13mm
{\tt 本人完全了解中国人民大学有关保留、使用学位论文的规定,即:学校有保留送交论文的复印件,允许论文被查阅和借阅;学校可以公布论文的全部或部分内容,可以采用影印、缩印或其他复制手段保存论文。\par
\vskip 10mm\hfill 
论文作者:\rule[-1ex]{30mm}{0.25pt}\qquad 
日\qquad 期:\rule[-1ex]{30mm}{0.25pt}\par
\hfill \rule[0mm]{0mm}{8mm}
指导老师:\rule[-1ex]{30mm}{0.25pt}\qquad 
日\qquad 期:\rule[-1ex]{30mm}{0.25pt}\par}
\thispagestyle{empty}
}
%    \end{macrocode}
% \end{macro}	
% \subsection{章节标题}
%     \begin{macrocode}
\setcounter{secnumdepth}{4}
\ifruc@bachelor
\ctexset{
	chapter = {
	name = {,},
	number = \arabic{chapter}
	},
	section = {
	number = \thechapter.\arabic{section}
	},
	subsection = {
	number = \thesection.\arabic{subsection}
	},
	subsubsection = {
	number = \thesubsection.\arabic{subsubsection}
	},
	chapter/format =\vspace{-15mm}\centering,
	chapter/numberformat = \bf\zihao{3},
	chapter/titleformat = \sf\zihao{3},
	chapter/nameformat = \sf\zihao{3},
	section/format = ,
	section/numberformat = \bf\zihao{4},
	section/titleformat = \sf\zihao{4},
	subsection/format = ,
	subsection/numberformat = \bf\zihao{-4},
	subsection/titleformat = \sf\zihao{-4},
	subsubsection/format = ,
	subsubsection/numberformat = \bf\zihao{5},
	subsubsection/titleformat =\sf \zihao{5}
}
\else
\ctexset{
	chapter = {
	name = {第,章},
	number = \arabic{chapter}
	},
	chapter/format =\centering,
	chapter/numberformat = \bf\zihao{-2},
	chapter/nameformat = \sf\zihao{-2},
	chapter/titleformat = \sf\zihao{-2},
	section/format = ,
	section/titleformat =\sf\zihao{-3},
	section/numberformat = \bf\zihao{-3},
	subsection/format = ,
	subsection/numberformat = \bf\zihao{-4},
	subsection/titleformat = \sf\zihao{-4},
	subsubsection/format = ,
	subsubsection/titleformat = \sf\zihao{5},
	subsubsection/numberformat = \bf\zihao{5}	
}
\fi
%    \end{macrocode}
% \subsection{目录样式}
%    \begin{macrocode}
\RequirePackage{titletoc}
\titlecontents{chapter}[0pt]{\sf\zihao{4}\addvspace{2pt}\filright}
	{\contentspush{\thecontentslabel\ }}
	{}{\titlerule*[10pt]{.}{\bf\contentspage}}
\titlecontents{section}[2em]{\rm\zihao{-4}\addvspace{2pt}\filright}
	{\contentspush{\thecontentslabel\ }}
	{}{\titlerule*[10pt]{.}\contentspage}
\titlecontents{subsection}[4em]{\rm\zihao{-4}\addvspace{2pt}\filright}
	{\contentspush{\thecontentslabel\ }}
	{}{\titlerule*[10pt]{.}\contentspage}
\titlecontents{figure}[10pt]{\rm\zihao{-4}\addvspace{2pt}}
	{图~\thecontentslabel\, }
	{}{\titlerule*[10pt]{.}\contentspage}
\titlecontents{table}[10pt]{ \rm\zihao{-4}\addvspace{2pt}}
	{表~\thecontentslabel\, }
	{}{\titlerule*[10pt]{.}\contentspage}     
%    \end{macrocode}   
% \subsection{本科脚注}
% \begin{macro}{\ruc@textcircled}  本科的圆形脚注...
%    \begin{macrocode}
\RequirePackage{ifxetex}
\RequirePackage{ifthen,calc}
\def\ruc@textcircled#1{%
\ifnum \value{#1} <10 \textcircled{\zihao{-6}\arabic{#1}}
\else\ifnum \value{#1} <100 \textcircled{\zihao{7}\arabic{#1}}\fi
\fi}
\ifruc@bachelor
\renewcommand{\thefootnote}{\ruc@textcircled{footnote}}
\renewcommand{\thempfootnote}{\ruc@textcircled{mpfootnote}}
\def\footnoterule{\vskip-3\p@\hrule\@width0.3\textwidth\@height0.4\p@\vskip2.6\p@}
\let\ruc@footnotesize\footnotesize
\renewcommand\footnotesize{\ruc@footnotesize\zihao{-5}}
\def\@makefnmark{\textsuperscript{\hbox{\normalfont\@thefnmark}}}
\long\def\@makefntext#1{
\bgroup
\newbox\ruc@tempboxa
\setbox\ruc@tempboxa\hbox{%
\hb@xt@ 1.5em{\@thefnmark\hss}}
\leftmargin\wd\ruc@tempboxa
\rightmargin\z@
\linewidth \columnwidth
\advance \linewidth -\leftmargin
\parshape \@ne \leftmargin \linewidth
\footnotesize
\@setpar{{\@@par}}%
\leavevmode
\llap{\box\ruc@tempboxa}%
#1
\par\egroup}\else\fi
%    \end{macrocode}
% \end{macro}
% \subsection{本科签名}
% \begin{macro}{\autograph} 插入签名,在普通章节文件中插入无限制,但在\tt main.tex\rm 文件插入时被插入的章节不能使用\tt\textbackslash include\rm 命令,需使用\tt\textbackslash input\rm 命令。
%    \begin{macrocode}
\ifruc@bachelor
\newcommand{\autograph}{  %
{\noindent\zihao{4}\sf\rule[0mm]{0mm}{15mm}作者签名:
\rule[-1ex]{30mm}{0.25pt}}}
\else\fi
%    \end{macrocode}
% \end{macro}
% \subsection{封面}
%  \noindent 封皮都是研究生院统一从印刷厂印制,普通的打印店貌似做不出来......
%    \begin{macrocode}
\RequirePackage{lscape}
\RequirePackage{multicol}
\RequirePackage{color}
\RequirePackage[dvipsnames,prologue,table]{pstricks}
\definecolor{rucblue}{rgb}{0.129,0.482,0.663}
\definecolor{rucorange}{rgb}{0.976,0.769,0.325}
\definecolor{rucred}{rgb}{0.569,0.129,0.2}
\definecolor{rucgreen}{rgb}{0.333,0.608,0.627}
\definecolor{rucwhite}{rgb}{0.999,0.999,0.999}
\definecolor{rucbalck}{rgb}{0,0,0}
%    \end{macrocode}
% \begin{macro}{\X}封面标识使用的是方正小标宋字体,PostScript名称:FZXBSJW--GB1-0
%    \begin{macrocode}
\setCJKfamilyfont{X}{FZXBSJW--GB1-0}
\newcommand{\X}{\CJKfamily{X}}
%    \end{macrocode}
% \end{macro}
% \begin{macro}{\cover}封面生成命令需在\tt cover.tex\rm 文件中使用
%    \begin{macrocode}
\newcommand{\cover}{
\newgeometry{top=0mm,left=20mm,bottom=0mm,right=0mm,voffset=10mm,columnsep=5.5cm}
\ifruc@doctor
\pagecolor{rucorange}
\else
\ifruc@master
\pagecolor{rucblue}
\else
\ifruc@promaster
\pagecolor{rucgreen}
\else
\pagecolor{rucred}
\fi
\fi
\fi
\begin{landscape}
\begin{multicols}{2}
\rule[0mm]{22mm}{0mm}
\ifruc@doctor
\includegraphics[width=5.7cm,height=1.14cm]{figures/logo.pdf}\\
\else
\ifruc@promaster
\includegraphics[width=6.3cm,height=1.26cm]{figures/logoW.pdf}\\
\else
\includegraphics[width=5.7cm,height=1.14cm]{figures/logoW.pdf}\\
\fi\fi
{\X
\ifruc@doctor
\color{rucred}
\else\color{rucwhite}
\fi\zihao{2}\rule[10mm]{30mm}{0mm}
\ifruc@doctor
\makebox[5.7cm][s]{\doctorsign}\\
\zihao{-4}\rule[6mm]{32mm}{0mm}\makebox[5.7cm][s]{\doctoresign}
\else
\ifruc@master
\makebox[5.7cm][s]{\mastersign}\\
\zihao{-4}\rule[6mm]{32mm}{0mm}\makebox[5.7cm][s]{\masteresign}
\else
\ifruc@promaster
\makebox[6.3cm][s]{\promastersign}\\
\zihao{-5}\rule[6mm]{32mm}{0mm}\makebox[6.3cm][s]{\promasteresign}
\else
\makebox[5.7cm][s]{\bachelorsign}\\
\zihao{5}\rule[0mm]{32mm}{0mm}\makebox[5.7cm][s]{\bacheloresign}
\fi\fi\fi}
\vfill
\begin{center}
\rule[0mm]{0mm}{30mm}\\
\ifruc@doctor
\includegraphics[width=2cm,height=2cm]{figures/logo2.pdf}\vspace{2mm}\\
\includegraphics[width=5.7cm,height=1.5048cm]{figures/logo3.pdf}\\
\else
\includegraphics[width=2cm,height=2cm]{figures/logoW2.pdf}\vspace{2mm}\\
\includegraphics[width=5.7cm,height=1.5048cm]{figures/logoW3.pdf}\\
\fi
{\X
\ifruc@doctor
\color{rucred}
\else
\color{rucwhite}
\fi
\zihao{0}\ziju{0.3}
\ifruc@doctor\doctorsign
\else
\ifruc@master\mastersign
\else
 \ifruc@promaster\ziju{0.1}\promastersign
\else\bachelorsign
\fi\fi\fi\par\zihao{-4}\rule[0mm]{0mm}{10mm}
\ifruc@doctor\doctoresign
\else
\ifruc@master\masteresign
\else
\ifruc@promaster\promasteresign
\else\bacheloresign
\fi\fi\fi\par}\sf\zihao{-3}
\vspace{40mm}\ziju{0}\linespread{1.5}\rule[0mm]{0mm}{30mm}
\ifruc@doctor\color{rucbalck}\else\color{rucwhite}\fi
论文题目:\underline{\parbox[b]{100mm}\@title}\hfill\par \rule[0mm]{0mm}{10mm}
英\qquad 文:\underline{\parbox[b]{100mm}\@etitle}\hfill\par
\vspace{20mm}\
作\qquad 者:\underline{\makebox[100mm]\@author}\hfill\par\rule[0mm]{0mm}{10mm}
指导教师:\underline{\makebox[100mm]\@advisor}\hfill\par
\vspace{18mm}\@date
\end{center}
\end{multicols}
\thispagestyle{empty}
%    \end{macrocode}
% \end{macro}
% 要求中只有博士是需要书脊的(硕士可选),需要设置纸张高度为两张A4纸宽度(420mm)再加书脊的宽度,100张A4纸厚度约为1cm。
%    \begin{macrocode}
\ifruc@shuji
\ifruc@doctor\color{rucbalck}\else\color{rucwhite}\fi
\uput[u]{0}(0.5\paperheight,15){{\sf\zihao{3}\shortstack[c]{\@covertitle}}}
\uput[u]{0}(0.5\paperheight,3){{\sf\zihao{3}\shortstack[c]{中\\国\\人\\民\\大\\学}}}
\else\fi
\end{landscape}
\restoregeometry
}
%    \end{macrocode}
% \subsection{参考文献}
%    \begin{macrocode}
\RequirePackage{natbib}%参考文献
%    \end{macrocode}
% \subsection{图表caption设定}
%    \begin{macrocode}
\RequirePackage{caption2}
\captionstyle{normal}
\renewcommand{\captionfont}{\zihao{5}\rm}
\renewcommand{\captionlabelfont}{\zihao{5}\bf}
\renewcommand{\captionlabeldelim}{\,\,}
%    \end{macrocode}
% \subsection{表格宏包}
%    \begin{macrocode}
\RequirePackage{booktabs}%三线表
\RequirePackage{colortbl}%表格颜色
\RequirePackage{diagbox}%表头制作
\RequirePackage{longtable}%长表格
\let\ruc@LT@array\LT@array
\def\LT@array{\zihao{5}\rm\ruc@LT@array}
\let\old@tabular\@tabular
\def\ruc@tabular{\old@tabular\zihao{5}}
\RequirePackage{multirow}%跨行宏包
%    \end{macrocode}
% \subsection{表格caption上下间距}
%    \begin{macrocode}
\setlength{\abovecaptionskip}{6pt}  
\setlength{\belowcaptionskip}{6pt} 
%    \end{macrocode}
% \subsection{插图宏包}
%    \begin{macrocode}
\RequirePackage{wrapfig}%图文混排,段落首字下沉
\RequirePackage{subfigure}%子图
\RequirePackage{tikz}%绘图
%    \end{macrocode}
% \subsection{数学宏包}
%    \begin{macrocode}
\RequirePackage{amsmath,amsfonts}%
\RequirePackage{latexsym,bm}%
\RequirePackage{extarrows}%
\RequirePackage{pifont}%
%</cls>
%    \end{macrocode}
% \Finale
 \endinput
