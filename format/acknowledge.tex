\begin{acknowledge}%致谢

那末,德国解放的实际可能性到底在哪里呢?
      答:就在于形成一个被彻底的锁链束缚着的阶级,即形成一个非市民社会阶级的市民社会阶级,一个表明一切等级解体的等级;一个由于自己受的普遍苦难而具有普遍性质的领域,这个阶级并不要求享有任何一种特殊权利,因为它的痛苦不是特殊的无权,而是一般无权,它不能再求助于历史权利,而只能求助于人权,它不是同德国国家制度的后果发生片面矛盾,而是同它的前提发生全面矛盾,最后,它是一个若不从其它一切社会领域解放出来并同时解放其它一切社会领域,就不能解放自己的领域,总之是这样一个领域,它本身表现了人的完全丧失,并因而只有通过人的完全恢复才能恢复自己。这个社会解体的结果,作为一个特殊的等级来说,就是无产阶级。
      德国无产阶级是随着刚刚着手为自己开辟道路的工业的发展而形成起来的;因为组成无产阶级的不是自发产生的而是人工制造的贫民,不是在社会的重担下机械地压出来的而是由于社会的急剧解体过程,特别是由于中间等级的解体而产生的群众,不言而喻,自发产生的贫民和基督教德意志的农奴等级也在不断地-虽然是逐渐地-充实无产阶级的队伍。
      无产阶级宣告现存世界制度的解体,只不过是揭示自己本身存在的秘密,因为它就是这个世界制度的实际解体。无产阶级要求否定私有财产,只不过是把社会已经提升为无产阶级的原则的东西,把未经无产阶级的协助,作为社会的否定结果而体现在它的身上,即无产阶级身上的东西提升为社会的原则。无产阶级对正在形成的世界所享有的权利和德国国王对已经形成的世界所享有的权利是一样的。德国国王把人民称为自己的人民,正像他把马叫作自己的马一样。国王宣布人民是他的私有财产,只不过表明私有财产的所有者就是国王这样一个事实。
      哲学把无产阶级当做是自己的物质武器,同样地,无产阶级也把哲学当作自己的精神武器;思想的闪电一旦真正射入这块没有触动过的人民园地,德国人就会解放为人。

根据上述一切,可以得出如下的结论:
      德国唯一实际可能的解放是从宣布人是人的最高本质这个理论出发的解放。在德国,只有从对中世纪的部分胜利解放出来,才能从中世纪得到解放。在德国,不消灭一切奴役制,任何一种奴役制都不可能消灭。彻底的德国不从根本上开始进行革命,就不可能完成革命。德国人的解放就是人的解放。这个解放的头脑是哲学,它的心脏是无产阶级。哲学不消灭无产阶级,就不可能成为现实;无产阶级不把哲学变成现实,就不可能消灭自己。

一切内在条件一旦成熟,德国的复活日就会由高卢雄鸡的高鸣来宣布。
\end{acknowledge}
