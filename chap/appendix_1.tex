\chapter{如何正确安装\LaTeX\ }

Noun–verb dependencies in various languages and their biological ana- logues. Part A) shows the sentence “Dick saw Jane help Mary draw pictures” trans- lated grammatically into German and Dutch. That is, the words in the sentence are rearranged to reflect the rules of grammar in these two languages, but the sentence is not translated per se. As shown, the English version of the sentence has a rela- tively simple dependency structure between the nouns and verbs that can be modeled using regular grammars. In contrast, German and Dutch require more complicated grammatical models . Part B) shows the biological analogue of the three sen- tences in Part A). Typically, restriction sites can be modeled using regular grammars, whereas complex DNA secondary structures require context–free or context–sensitive grammars . In the first example, the arches are used to represent a “must be followed by” dependency. In the second two examples, they represent a “must be complementary to” dependency.
